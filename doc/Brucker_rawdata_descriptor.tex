\documentclass[a4paper]{article}

% for extended summary
%\usepackage[summary]{nnsp2e}
% final paper
\usepackage{nnsp2e}
\usepackage[latin1]{inputenc}
\usepackage[francais]{babel}
\usepackage{amssymb}%enumerate,theorem,setspace
\usepackage{shortvrb,subeqnarray}
\usepackage{color}
\usepackage{array}

% Example definitions.
% --------------------
\input{alphabet}   
\input{abrege}	
\input{abrmath}  
%\input{old/abrege}	
%\input{old/abrmath}  
\input{beginend}   
\input{transdef}
\title{Description des donn�es brutes sur l'IRM 3T Bruker}

\author{Cyril Poupon, \today}
%\author{Jan Larsen and Cyril Goutte\\
%        Department of Mathematical Modeling, Building 321\\
%        Technical University of Denmark, DK-2800 Lyngby, Denmark\\
%        Phone: +45 4525 3923,3921\\
%        Fax: +45 4587 2599\\
%        E-mail: jl,cg@imm.dtu.dk\\
%        Web: eivind.imm.dtu.dk}


\begin{document}
\maketitle

Voici la mani�re dont je suppose organiser les r�pertoires des donn�es
brutes et reconstruites sur l'IRM 3T \textbf{Bruker} (ca ne veut pas dire qu'il n'y a a pas  d'autres fichiers dedans, mais je n'en tiens pas compte):

\section{Diff�rentes organisations des donn�es}

Dans la suite \texttt{<XXXXXX>} indique le r�pertoire dont le nom est typiquement \texttt{bruXXXX} 

\subsection{Donn�es brutes Paravision 1.0}

\bcc
\btabu{rcl}
\texttt{<XXXXXX>/} && nom du r�pertoire\\
\texttt{<XXXXXX>/acqp} &       $\longrightarrow$    & contient les params d'acquisition bas niveau\\
\texttt{<XXXXXX>/imnd} &       $\longrightarrow$    & contient les params d'acquistion haut niveau\\
\texttt{<XXXXXX>/fid}  &        $\longrightarrow$   & contient les donn�es complexes du $k$-space
\etabu
\ecc

\subsection{Donn�es brutes Paravision 3.0}
\bcc
\btabu{rcl}

\texttt{<XXXXXX>/}&&nom du r�pertoire \\
\texttt{<XXXXXX>/acqp}&       $\longrightarrow$    & contient les param d'acquisition bas niveau\\
\texttt{<XXXXXX>/method }   &$\longrightarrow$&contient les params d'acquistion haut niveau (idem imnd)\\
\texttt{<XXXXXX>/fid}&&
\etabu
\ecc

\subsection{Donn�es reconstruites Paravision 1.0}
\bcc
\btabu{rcl}
\texttt{<XXXXXX>/}&&\\
\texttt{<XXXXXX>/reco}       &$\longrightarrow$&contient les params de la reconstruction\\
\texttt{<XXXXXX>/../../acqp}&&\\
\texttt{<XXXXXX>/../../imnd}&&\\
\texttt{<XXXXXX>/2dseq}     &$\longrightarrow$&contient les donn�es r�elles reconstruites
\etabu
\ecc

\subsection{Donn�es reconstruites Paravision 3.0}
\bcc
\btabu{rcl}
\texttt{<XXXXXX>/} && nom du r�pertoire\\
\texttt{<XXXXXX>/reco} &$\longrightarrow$&contient les params de la reconstruction \\
\texttt{<XXXXXX>/../../acqp} &       $\longrightarrow$    & contient les params d'acquisition bas niveau \\
\texttt{<XXXXXX>/../../method} &$\longrightarrow$&contient les params d'acquistion haut niveau\\
\texttt{<XXXXXX>/2dseq} &$\longrightarrow$&contient les donnees r�elles reconstruites \\
\etabu
\ecc

\section{Structure des r�pertoires � l'acquisition}

En fait, � l'acquisition, un r�pertoire non modifi� Bruker a la t�te suivante
(je n'ai repr�sent� que les fichiers utiles... d'autres fichiers peuvent �tre
pr�sents dans cette arborescence, mais sont cr�es lors de post-processings
r�alis�s sur la station Paravision, et ne nous servent pas, comme par
exemple des fichiers ROI qui decrivent des r�gions d'interet utilis�es par
l'outil Bruker de stat sur ROIs,....):

\textbf{Paravision 1.0:}
%\begin{verbatim}
\bcc
\btabu{rcl}
\texttt{<YYYYYY>/1/} & $\longrightarrow$ & serie 1 de l'exam \texttt{YYYYYY} \\
\texttt{<YYYYYY>/1/imnd}        && \\              
\texttt{<YYYYYY>/1/acqp}  && \\
\texttt{<YYYYYY>/1/imnd}  && \\
\texttt{<YYYYYY>/1/fid}   && \\
\texttt{<YYYYYY>/1/pdata/}             &$\longrightarrow$ & reconstruction de la serie 1\\
\texttt{<YYYYYY>/1/pdata/1/}          & $\longrightarrow$&  reconstruction \Numero 1 de la serie 1\\
\texttt{<YYYYYY>/1/pdata/1/reco} && \\
\texttt{<YYYYYY>/1/pdata/1/2dseq}&&  \\
\texttt{<YYYYYY>/1/pdata/2/}          &$\longrightarrow$& reconstruction \Numero 2 de la serie 1 \\
\texttt{<YYYYYY>/1/pdata/2/reco} &&  \\
\texttt{<YYYYYY>/1/pdata/2/2dseq}&& \\
\texttt{<YYYYYY>/2/}                     &$\longrightarrow$ & serie 2 de l'exam \texttt{YYYYYY} \\
\texttt{<YYYYYY>/2/acqp}  &&  \\
\texttt{<YYYYYY>/2/imnd}  &&  \\
\texttt{<YYYYYY>/2/fid} &&    \\
\texttt{<YYYYYY>/2/pdata/}             &$\longrightarrow$& reconstruction de la serie 2\\
\texttt{<YYYYYY>/2/pdata/1/}          &$\longrightarrow$& reconstruction \Numero 1 de la serie 2\\
\texttt{<YYYYYY>/2/pdata/1/reco}  &&  \\
\texttt{<YYYYYY>/2/pdata/1/2dseq} &&  \\
\texttt{<YYYYYY>/2/pdata/2/}         & $\longrightarrow$& reconstruction \Numero 2 de la serie 2\\
\texttt{<YYYYYY>/2/pdata/2/reco}  &&\\
\texttt{<YYYYYY>/2/pdata/2/2dseq} &&\\
\texttt{<YYYYYY>/2/pdata/3/ }         &$\longrightarrow$& reconstruction \Numero 3 de la serie 2\\
\texttt{<YYYYYY>/2/pdata/3/reco}  && \\
\texttt{<YYYYYY>/2/pdata/3/2dseq} &&
\etabu
\ecc
%\end{verbatim}

\subsection{Diff�rentes reconstructions sur les m�mes donn�es}

Comme je l'ai d�j� �voqu�, des m�mes donn�es brutes peuvent �tre
reconstruites avec des param�tres de reconstruction differents, d'o�
l'existence possible de plusieurs recontructions pour une serie de
donn�es brutes (� titre d'exemple, des donn�es acquises avec une
matrice du $k$-space de 64x64 peuvent tres bien �tre reconstruites
avec une matrice de 64x64 ou 128x128 en faisant du zero-padding)

\edoc
